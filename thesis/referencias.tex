\bibliographystyle{plainnat}
\begin{thebibliography}{20}
\addcontentsline{toc}{chapter}{Refer�ncias}
\begin{small}

  \bibitem[Amante e Eakins (2009)]{amante&eakins2009} AMANTE, C.; EAKINS, B.W. ETOPO1 1 Arc-Minute Global Relief Model: Procedures, Data Sources and Analysis. \textbf{NOAA Technical Memorandum NESDIS NGDC-24}, p. 19, 2009.

  \bibitem[Asgharzadeh \textit{et al.} (2007)]{asgharzadeh_etal2007} ASGHARZADEH, M.F.; VON FRESE, R.R.B.; KIM, H.R.; LEFTWICH, T.E.; KIM, J.W. Spherical prism gravity ef\mbox{}fects by Gauss-Legendre quadrature integration. \textbf{Geophysics Journal International}, v. 169, p. 1-11, 2007.

  \bibitem[Barrera-Figueroa \textit{et al.} (2006)]{barrera-figueroa_etal2006} BARRERA-FIGUEROA, V.; SOSA-PEDROZA, J.; L�PEZ-BONILLA, J. Multiple root finder algorithm for Legendre and Chebyshev polynomials via Newton's method. \textbf{Annales Mathematicae et Informaticae}, v. 33, p. 3 - 13, 2006.
 
  \bibitem[Heck e Seitz (2007)]{heck&seitz2007}HECK, B.; SEITZ, K. A comparison of the tesseroid, prism and point-mass approaches for mass reductions in gravity field modelling. \textbf{Journal of Geodesy}, v. 81, p. 121 - 136, 2007.

  \bibitem[Heiskanen and Moritz (1967)]{h&m1967} HEISKANEN, W.A.; MORITZ, H. \textbf{Physical Geodesy}. W. H. Freeman and Company, San Francisco, 1967.

  \bibitem[Hildebrand (1987)]{hildebrand1987} HILDEBRAND. F.B.  \textbf{Introduction to numerical analysis}. Courier Dover Publications, 2. ed., 1987.

  \bibitem[Ku (1977)]{ku1977} KU, C.C. A direct computation of gravity and magnetic anomalies caused by 2- and 3-dimensional bodies of arbitrary shape and arbitrary magnetic polarization by equivalent-point methot and a simplified cubic spline. \textbf{Geophysics}, v. 42, p. 610 - 622, 1977.

%   \bibitem[Makhloof e Ilk (2008)]{makhloof&ilk2008} MAKHLOOF, A.A.; ILK, K. Effects of topographic-isostatic masses on gravitational functionals at the Earth's surface and at airborne and satellite altitudes. \textbf{Journal of Geodesy}, v. 82, p. 93 - 111, 2008.

%   \bibitem[Nagy (1966)]{nagy1966} NAGY, D. The gravitational attraction of a right rectangular prism. \textbf{Geophysics}, v. 31, p. 362 - 371, 1966.


  \bibitem[Nagy (2000)]{nagy2000} NAGY, D.; PAPP, G.; BENEDEK, J. The gravitational potential and its derivatives for the prism. \textbf{Journal of Geodesy}, v. 74, p. 552 - 560, 2000.


  \bibitem[Press \textit{et al.} (1992)]{numericalrecipes} PRESS, W.H.; FLANNERY, B.P.; TEUKOLSKY, S.A.; VETTERLING, W.T. \textbf{Numerical Recipes in C: The Art of Scientific Computing}. Cambridge University Press, 2. ed., 1992.

  \bibitem[Smith, Robertson e Milbert (2001)]{smith_etal2001} SMITH, D.A.; ROBERTSON, D.S.; MILBERT, D.G. Gravitational attraction of local crustal masses in spherical coordinates. \textbf{Journal of Geodesy}, v. 74, p. 783 - 795, 2001.

  \bibitem[Tscherning (1976)]{tscherning1976} TSCHERNING, C.C. Computation of the second-order derivatives of the normal potential based on the representation by a Legendre series. \textbf{Manuscripta Geodaetica}, v. 1, p. 71 - 92, 1976.

  \bibitem[Van�\v{c}ek e Krakiwsky (1986)]{vanicek1986} VAN�\v{C}EK, P.; KRAKIWSKY, E. \textbf{Geodesy: The Concepts}. Elsevier Science Publishers B.V., 2. ed., 1986.

  \bibitem[Wild-Pfeiffer (2008)]{wild-pfeiffer2008} WILD-PFEIFFER, F. A comparison of different mass elements for use in gravity gradiometry. \textbf{Journal of Geodesy}, v. 82 (10), p. 637 - 653, 2008.
\end{small}
\end{thebibliography}

