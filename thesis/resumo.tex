\begin{center}
\section*{Resumo}
\addcontentsline{toc}{chapter}{Resumo}
\end{center}

A miss�o de sat�lite GOCE tem o objetivo de medir o campo gravitacional da Terra
com acur�cia sem precedentes atrav�s de medi��es do tensor gradiente da
gravidade (TGG).
Os dados provenientes desta miss�o poder�o ser utilizados para estudar �reas
extensas, onde considera��es de Terra plana podem apresentar limita��es.
Para levar em conta a curvatura da Terra a modelagem pode ser feita utilizando
tesser�ides, tamb�m chamados de prismas esf�ricos.
O TGG causado por um tesser�ide pode ser calculado utilizando m�todos num�ricos
de integra��o, como por exemplo, a Quadratura Gauss-Legendre (QGL).
Neste trabalho foi implementado um programa computacional para o c�lculo direto
do TGG utilizando a QGL.
A precis�o desta implementa��o foi avaliada comparando seus resultados com o
resultado de f�rmulas anal�ticas para o caso especial de uma casca esf�rica.
Em seguida, o programa desenvolvido foi utilizado para calcular as diferen�as
causadas no TGG pela aproxima��o plana para a Terra.
Estas diferen�as s�o de at� 30\% na componente $T_{zz}$ para um modelo de
$50^\circ \times 50^\circ \times 10\ km$.
Por fim,  o programa desenvolvido foi utilizado para calcular o efeito das
massas topogr�ficas no TGG a 250 km de altitude para a regi�o da bacia do
Paran�.
Em regi�es de varia��es expressivas da topografia, as amplitudes das
componentes do TGG devido �s massas topogr�ficas possuem a mesma ordem de
grandeza do efeito no TGG causado por anomalias de densidade no interior da
crosta e manto.
