\begin{center}
\section*{Abstract}
\addcontentsline{toc}{chapter}{Abstract}
\end{center}


The GOCE satellite mission has the objective of measuring the Earths
gravitational field with an unprecedented accuracy through the measurement of
the gravity gradient tensor (GGT).
The data provided by this mission could be used to study large areas, where the
flat Earth approximation can have its limitations.
In these cases the modeling could be done with tesseroids, also called spherical
prisms, in order to take the Earths curvature into account.
The GGT caused by a tesseroid can be calculated with numerical integration
methods, such as the Gauss-Legendre Quadrature (GLQ).
In the current project, a computer program was developed for the direct
calculation of the GGT using the GLQ.
The accuracy of this implementation was evaluated by comparing its results with
the result of analytical formulas for the special case of a spherical cap.
Next, the developed program was used to calculate the differences in the GGT
caused by the flat Earth approximation.
These differences reach are up to 30\% in the $T_{zz}$ component for a
$50^\circ \times 50^\circ \times 10\ km$ model.
Finally, the computer program was used to calculate the effect caused by the
topographic masses on the GGT at 250 km altitude for the Paran� basin region.
In regions of large topographical variations, the components of the GGT due to
the topographic masses have amplitudes of the same order of magnitude as the GGT
components due to density anomalies in the interior of the crust and mantle.